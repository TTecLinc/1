\documentclass{article}
\usepackage{caption}
\usepackage{mathrsfs}
\usepackage{amsmath}
\usepackage{amsfonts}
\usepackage{extarrows}
\usepackage{mwe}
\newtheorem{remark}{Remark}[section]
\begin{document}
\section{Reference}
\section{Empirical Articles}

\subsection{Social proximity and firm performance: the importance of family member ties in workplaces}
    ~\

    This article assesses the role of social proximity, i.e. the concentration of family members in firms.
    
    The innovation of this article is: it addresses the relation betweeen family linkages and regional differences in frim productivity
    in a more systematic way. It shows that the impact of social proximity on firm performance differ
    depending on the spatial context and the access to agglomeration externalities.

    In terms of data, the dependent variable is labour productivity defined as per capita value-added. The main independent variable 
    is a proxy for social proximity(FM). The control variables include a number of firm-level characteristics:
    
    \emph{1. high education(human capital)} 
    
    \emph{2. firm size(employment)} 
    
    \emph{3. capital(the depreciation on fixed assets as a proxy for capital). }
    
    \emph{4. regional size: total number of employees by region and year(N/100)}
    
    \emph{5. specialization: LQ}

    In order to control the effects of agglomeration economies with two measures. One is location quotient(LQ), 
    \begin{equation}
        SP_{ir}=\frac{e_{ir}/e_r}{e_i/e}
    \end{equation}
    the other is using log regional employment size. 

    $SP_{ir}$ is the degree of specialization in industry $i$ in region $r$, 
    $e_{ir}$ is the number of employees in two-digit industry $i$ in region $r$, 

    The model is:
    \begin{equation}
        lnY_i=\beta_0+\sum_{i=1}^n\beta_1X_i+\sum_{i=1}^n\delta_2A_i+\epsilon
    \end{equation}
    $lnY_i$ represents average labour productivity in firms $i$, 
    $\sum_{i=1}^n\beta_1X_i$ represents the sum of different family ties and $\sum_{i=1}^n\delta_2A_i$ represents the sum of control variables.
    
    The results indicate that:

    \emph{1. Concentration of FM positively affect performance, including the jointly regressed with both firm and regional level factors. (Especially for smaller firms)}

    \emph{2. FM significantly influences performance in low technological manufacturing firms with a relatively low educated workforce. }

    \emph{3. Social proximity should be less important in large specialized regions because of the presence of specialzied skills.}

    From the perspective of robustness, he performed four main robustness checks: estimating the marginal effects 
    for each variable at the mean of the other variables on model 1 (FM only), 
    model 3 (FM + controllers), model 8 (OLS with all interactions) and model 9 (FE with all interactions). 
    It is verified that FM is an attribute that mainly affects the performance of small companies.


\subsection{Familial relationships and firm performance: the impact of entrepreneurial family relationships}
    ~\

    The aim of the present paper is to bring further clarity to the question of what types of family relationships within the firm may positively influence performance. 
    This article argue that different forms of relationships in the entrepreneur's family (children, spouse and siblings) are particularly strong forms of social relations that can transform the 
    influence of cognitive proximity in the firm.

    They propose three hypotheses: 
    
    \emph{1. Strong entrepreneur–children relationships (co-occurrence) in the firm will mitigate/ lower the negative effects of similar and very different skill sets, and enhance the positive effects of related skills and human capital on productivity.}
    
    \emph{2. Strong entrepreneur–sibling relationships (co-occurrence) in the firm will mitigate/ lower the negative effects of similar and very different skill sets and enhance the positive effects of related skills and human capital on productivity.}

    \emph{3. Strong entrepreneur–partner relationships (co-occurrence) in the firm will mitigate/ lower the negative effects of similar and very different skill sets and enhance the positive effects of related skills and human capital on productivity.}

    Dependent variable is log labour productivity.
    Independent variable is family relations(counts of entrepreneur's children in the same firm), including entrepreneur–children, entrepreneur-partner, entrepreneur-sibling. 
    Control variables: 
    
    \emph{1. SIM, degree of similarity of formal education.}

    \emph{2. REL, relatedness of formal education.}

    \emph{3. UNREL, unrelatedness of formal education. }

    \emph{4. Frim size}

    \emph{5. Higher education: share of workers with a minimum of Bachelor's degree.}

    \emph{6. Capital}

    \emph{7. Regional size: Employment by region}

    \emph{8. Specialization: LQ}

    SIM is proposed by Boschma(2009), 
    \begin{equation}
        SIM=\frac{1}{\sum_{i=1}^{N^3}P_i^3log_2[1/P_i^3]}
    \end{equation}
    REL can be caculated similarly, 
    \begin{equation}
        REL=\sum_{j=1}^{N^2}P_j^2H_j=\sum_{j=1}^{N^2}\sum_{i\in S_i^2}P_i^3\sum_{i=S_i^2}\frac{P_i^3}{P_j^2}\bigg[\frac{1}{P_i^3/P_j^2}\bigg]
    \end{equation}
    UNREL is proposed by Frenken(2007),
    \begin{equation}
        UNREL=\sum_{l=1}^{N^1}P_i^1log_2[1/P_i^1]
    \end{equation}

    The model is:
    \begin{equation}
        \begin{aligned}
            lnP_{it}=&\beta_0+\beta_1[Entrepreneur's\; family\; relations_{it-1}]  
            +\beta_2[Control_{it}]+\\
            &\beta_3[Entrepreneur's\; family\; relations_{it}\times Skill\; variety_i]
            +v_i+\epsilon_{it}
        \end{aligned}
    \end{equation}

    We can partly confrim Hypothesis 1, family relationships involving entrepreneur-children 
    abate the negative effects of similarity in formal education (SIM) on productivity. 
    The significance of the entrepreneur–children variable disappears, which indicates that the 
    positive and significant estimate of Model 1 and 4 is mainly found in firms where the children have an educational 
    level similar to that of their parent entrepreneur.

    There is no support for Hypothesis 2, neither the entrepreneur-sibling variable nor any of the interactions are significant in any model. 

    Hypothesis 3 can be partly confirmed, because family relationships involving entrepreneur-partner are positively correlated with related competencies (REL), as well as mitigating the negative impact of similar and unrelated competences.
    However, t has a somewhat weaker association with productivity than the impact of related skills alone.

\subsection{Founding-Family Ownership and Firm Performance: Evidence from the SP 500}
    ~\

    The paper explore the relation between founding-family ownership and firm performance in large public firms. 
    They conduct :

    \emph{1. Time-series cross-sectional comparsion of family and nonfamily firms. }

    \emph{2. Investigating the association between active family control and firm performance. }

    \emph{3. The impact of other large equity blockholders in the presence of family ownership and the discrepency between 
    family ownership and control rights on firm performance. }

    The paper argue the disadvantages and advantages.

    \emph{Disadvantages}

    \emph{1. The family’s role in selecting managers and directors can also create impediments for third parties in capturing 
    control of the firm, suggesting greater managerial entrenchment and lower firm values relative to nonfamily firms. }

    \emph{2. Families are also capable of expropriating wealth from the firm through excessive compensation, related-party transactions, or special dividends.
    They are no longer competent or quali¢ed to run the firm.}

    \emph{3.  Families have incentives to redistribute rents from employees to themselves.}

    \emph{Advantages}

    \emph{1. The family’s wealth is so closely linked to firm welfare, families may have strong incentives 
    to monitor managers and minimize the freerider problem inherent with small, atomistic shareholders. }

    \emph{2. Firms that have shareholders with longer investment horizons suffer less managerial myopia and are therefore less likely to forgo good investments to boost current earnings.}

    \emph{3. One consequence of families maintaining a long-term presence is that the firm will enjoy a lower cost of debt financing compared to nonfamily firms.}

    The dependent variable is firm performance(ROA based on EBITDA and net income), independent variable
    is a binary variable that equals one when the founding family is present in the firm, and zero otherwise.
    control variables:
    \emph{officer and director holdings less family holdings, unaffiliated blockholdings, fraction of independent directors serving on the board, 
    fraction of total pay that the CEO receives in equity-based forms, research and development expenses divided by total sales, long-term debt divided by total assets, 
    stock return volatility, natural log of total assets, and the natural log of firm age;
    }

    The model is:
    \begin{equation}
        \begin{aligned}
            Firm \; Performance&=\delta_0+\delta_1(Family\; Firm)+\\
            &\delta_3(control\; variables)+\delta_{3-54}(Two\; digit\; SIC\; Code)+ \\
            &\delta_{93-99}(Year\; Dummy\; Variables)+\epsilon
        \end{aligned}
    \end{equation}

    There are conclusions:

    \emph{1. Younger firms have a greater impact, but both young and old family firms exhibit a significant and positive association to ROA, regardless of firm age.
    (Tobin's q in family firms is 10.0 percent higher than in nonfamily firms.)}

    \emph{2. Based on accouny=ting performance, family firms appear to be better performers only when a family member serves as CEO.}

    \emph{3. The founder descendants are unrelated to market performance.}

    \emph{4. The relation between firm performance and founding family ownership is nonlinear.}

\subsection{How do family ownership, control and management affect firm value?}
    ~\
    
    This paper try to understand whether and when family firms trade at a premium or discount relative to nonfamily firms.
    Three questions are solved: 

    \emph{1. Does family ownership per se create or destroy value?}

    \emph{2. Does family control create or destroy value?}

    \emph{3. Does family management create or destroy value?}

    Dependent variable: Tobin's q or Industry-adjusted q
    
    Independent variable: Family ownership 
    
    Control variable: \emph{Asset, Sales, Employees, Firm age since founding, generation, sales growth,
    ROA, Control-enhancing mechanisms, Family excess voteholdings, Governance index, Nonfamily blockholdings, Nonfamily outside directors, Dividends/book equity, Debt/market value of equity,
    Market risk (beta), Idiosyncratic risk, Diversification dummy, R\&D/sales, CAPX/PPE}   

    Conclusions:

    \emph{1. Despite the costs associated with the family's excess control, the benefits 
    of family ownership make minority shareholders better off than they would have been in
    a nonfamily firm.}

    \emph{2. The effect of blockholders is significant more negative for nonfamily 
    firms than it is for family firms, and blockholders are more likely to take control of 
    underperforming firms, especially those that are not already controlled by a family.}
    
    \emph{3. An absence of control-enhancing mechanisms curbs the family’s power to expropriate minority 
    shareholders and thus reduces the price that families pay for control. }

    \emph{4. If there are multiple share classes and the family’s super-voting shares trade, or are valued, 
    at a premium relative to the other class, minority shareholders pay a disproportionate share of the price for the founder’s control.
    The presence of control-enhancing mechanisms is not so much an indication of the desire to expropriate 
    minority shareholders as it is of family resistance to the dilution of their ownership stake as the firm grows.}
    

    The results are robust to the use of multivariate regression of $q$ and industry adjusted $q$. 
\subsection{Mixing family with business: A study of Thai business groups and the families behind them}
    ~\

    There are three main finding:
    
    \emph{1. Documenting in detail how control, managemnet and ownership are allocated 
    are allocated accross different family members. }

    \emph{2. Showing that families where the founder has a relatively greater number of sons are associated with lower firm-level performance.}

    \emph{3. Families that have relatively more sons tend to show a larger discrepancy between control and ownership rights (excess control.
    Moreover, sons show higher levels of excess control once the founder is gone. }

    In order to analysis the role of individual family members in the performance and 
    governmence of group firms, two types of involvement should be focus on: board membership and share ownership.
    
    Ownership structure: 
    
    \emph{They find a strong positive correlation between the size of the family and the number of family members who are involved in business. }

    Board positions: 

    \emph{1. The involvement of family members on the boards of firms increases with the size of the family, 
    and there is no significant change in the overall number of board positions when the founder is dead. }

    \emph{2. The fraction of board positions held by the sons is not significantly higher when the founder is dead.
    And the number of other family members has a negative and significant (albeit small) relation with the fraction of 
    board positions held by the sons. }

\section{Theory Articles}
\subsection{Blood and Money: Kin altruism, governance, and inheritance in the family firm}
    ~\

    Model:
    
    \emph{Preferences}

    \begin{equation}
        u^{Self}=v^{Self}+hv^{Relative}
    \end{equation}

    \emph{Effort}

    The distribution of cash flow $\tilde{x}$ meet: 
    \begin{equation}
        \tilde{x}=dist.\begin{aligned}
            \begin{cases}
                \tilde{x},\;&w.p.\;p \\
                0\;&w.p.\;1-p
            \end{cases}
        \end{aligned}
    \end{equation}
    Then the non-pecuniary effort cost is $\epsilon(p)$, it is a weakly increasing function of $p$.
   
    The expected cash flow to the project excceeds the cost of effort, monitoring, and the manager's reservation payoff:
    \begin{equation}
        \max_{p\in[0,\bar{p}]} p\bar{x}-(v_R+\epsilon(p)+c)>0
    \end{equation}
    The owner's utility benefit from monitoring when owner knows that the manager has underreported cash flow, 
    which equals the utility from transferring a concealed cash flow of $\tilde{x}$ from 
    the manager to the owner, $(1-h)\tilde{x}$ excceeds the cost of monitoring $c$.
    \begin{equation}
        (1-h)\bar{x}-c>0
    \end{equation}
    The first part is discussing kinship and monitoring when compensation and output are fixed, including:
    \emph{Incentives to underreport, Incentives to monitor, Monitoring/reporting equilibrium.}
    \begin{equation}
        \frac{PM'(h)}{PM}=\frac{c}{(1-h)[(1-h)\tilde{x}-c]}+\bigg(-\frac{c}{(1-h)[(1-h)\tilde{x}+ch]}\bigg)
    \end{equation}
    This same elasticity effect implies that $PM'(h)/PM$ is increasing, i.e., the probability of monitoring, 
    and thus monitoring expense, is log-convex in kinship. Log convexity implies that kinship’s marginal 
    effect on the monitoring problem is much greater when the owner and manager are close relatives, e.g., 
    both children of the founder, than when they are distant relations.
    
    The second part is when the frontier is open, 
    The thrid frontier is closed.
     
    \emph{1. Markets for human capital when management skills are firm specific}
    
    The effect of kinship on the manager’s value function is somewhat subtle. Recall, that the labor market participation constraint is always binding at the equilibrium compensation con- tract if the limited liability constraint is satisfied. However, this condition only ensures that the manager’s utility from accepting employment is constant not that the payoff from accepting employment is constant. Utility incorporates indirect internalized family gains and direct pay- off gains. Increasing kinship from a sufficiently high starting point can actually increase the payoff required to meet the manager’s participation constraint. This perhaps counterintuitive effect results because an increase in kinship increases monitoring expense and thus lowers fam- ily value. Hence, at higher kinship levels, the manager has less family gain to internalize. In order to keep the manager’s utility constant, the manager’s direct payoff gains must increase.
    
    \emph{2. Markets for human capital when management skills are general}
    
    At sufficiently high levels of human capital generality, the owner prefers to 
    hire the external manager. In which case, the kin manager works outside the family firm. Because the kin manager 
    is working outside the family firm, further increases in human capital generality do not affect the kin manager’s payoff. 
    However, further increases in generality, through their output increasing effect, increase owner payoffs.
    
    There are some extensions.

    \emph{1. Nepotism and rival internal managers}

    Inefficiency and nepotism are generated in the internal labor market framework because managers earn agency rents and owners prefer to keep rents “in the family.” In practice, one potential mechanism for mitigating these costs is to bring the internal rival into the family through marriage. Affine relations have offspring who are genetically related. Thus, if agents’ actions primarily affect the fitness of their descendants, the conditions for kin altruism are satisfied by affinity bonds.

    \emph{2. Founders vs. Descendants}

    In the agency setting and the labor market setting when the limited liability constraint binds, the founder has an incentive to design complex 
    mechanisms aimed at entrenching and increasing the compensation of the managing relative, N. A number of mechanisms might be employed to achieve this goal.


    


\subsection{Financing from Family and Friends}
    ~\

    The family finance is cheap, but it comes with shadow costs. The aspects that make informal finance less attractive including 
    greater risk aversion of informal investors, monitoring costs, social penalties, which imply a preium on required return. 

    The model in this paper account for both negative required returns and shadow costs. It is different
    from standard moral hazard model of external finance, where an entrepreneur can approcah a family and an outsider.  

    This article mainly discusses two issues: financing under altruism, financing and social debt.

    Altruism is characterized by a joint utility function, and the utility function is piecewise. 
    \begin{equation}
        U_E=(1-\phi)u(x_E)+\phi u(x_F)
    \end{equation}
    \begin{equation}
        u(x)=\begin{cases}
            x& for\; x\leq \underline{x}  \\

            \underline{x}+m(x-\underline{x})& otherwise
        \end{cases}
    \end{equation}
    For the financing problem under altruism, the article mainly carries on according to the familial insurance condition as the main line.
   
    \begin{equation}
        \frac{1}{m}>\frac{1-\phi}{\phi}
    \end{equation}
    If the familial insurance condition is violated, then two results can be conducted. 
    
    \emph{1. When the entrepreneur is unconstrained, altruism is irrelevant to the financing results.} 
    
    The first part reflects that $E$ prefers repaying $F$ rather than $O$ when the project succeeds, 
    while conversely, the second part reflects that $E$ prefers losing $O$’s rather than $F$’s wealth when the project fails. 
    Given $R_i = I_i/q$ for $i = F,O$, these components are equivalent. 
    \begin{equation}
        \begin{aligned}
            (1-\phi)\big\{E[u(I_{ext}-I+\tilde{R}-\tilde{R}_{ext})]-
            E[u(I_{ext}-&I+\tilde{B})]\big\}\ge \\
            \phi\big\{u(W-I_F-\phi E[u(W-I_F+\tilde{R}_F)])\big\} 
        \end{aligned}
    \end{equation}

    \emph{2. When the entrepreneur is risk- or capital-constrained, he prefers to use family finance. If the risk allocation is efficient, the source of funding is irrelevant.}


    If altruism is strong enough to satisfy the familial insurance condition, i.e. $\phi$ is high enough, outside finance will be more attractive. 

    \emph{1. When the entrepreneur is unconstrained, though he may be risk-constrained, outside finance dominates family finance. }
    
    The article prove that if the familial insurance condition is satisfied and the entrepreneur is capital-unconstrained. Then the entrepreneur prefers outside finance.
    If 
    \begin{equation}
        W-I<2\underline{x}<W-I+R
    \end{equation}
    the entrepreneur raises outside finance up to a threshold, but is otherwise indifferent between
    the financing sources. Both investors demand the same interest rate. The familial insurance condition 
    implies that $F$, when wealthy, is willing to insure $E$ against low consumption. That provided, constrained insurance condition implies that undertaking the project without outside finance 
    reduces $F$ ’s capacity to provide such insurance.

    \emph{2. When the entrepreneur is constrained, he prefers to finance from outside. }

    Another problem is social debt and project moral hazard. This part is expanded by the gift exchange condition: 
    \begin{equation}
        R_F\leq \frac{a^{+}-a^-}{c-a^+}\frac{\lambda}{1+\lambda}(W-I)
    \end{equation}

    If optimal financing is without project moral hazard, the condition is satisfied:
    First, for $c=1$, the interest rate equals the default premium under outside finance. Second, 
    the interest rate increases with $c$. But due to the lack of a default premium, a loan from $F$ is priced below the interest rate charged by $O$ for all $c$ that satisfy Assumption, that is, all favors that $F$ is willing to accept.

    If optimal financing is with project moral hazard: Suppose the entrepreneur is constrained and the gift exchange condition is satisfied. 
    There are $\{a^+, a^-, \lambda, c\}$ for which the project is feasible. If the project is feasible, the entrepreneur raises part of the financing from a friend as a donation or as a 
    loan at a below-market interest rate, and for $c > 1$, the remaining part through outside finance.
    It implies that a gift exchange is sometimes necessary for the project to be realized. There also exist parameter 
    constellations for which outside finance is necessary. To see this, consider E and F’s joint expected utility assuming their relationship is unharmed: $U_E + U_F = (1 + a^+) (u_E + u_F )$. In the absence of outside finance, the two friends are jointly better off 
    from undertaking the project only if
    \begin{equation}
        (1 + a^+) [W + qR - I - (1 - q)(c - 1)R_F^{**}]\ge (1 + a^+) W
    \end{equation}

\end{document}
